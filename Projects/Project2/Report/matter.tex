% !TeX root = Report.tex
% !TeX spellcheck = en_GB

\section{Approach}
We decided to build up the BDD tree to begin with, to allow for easy insertion of queen. 

\subsection{BDD construction}
The approach to building the BDD (\textit{binary decision diagram}) goes in several stages.\\
A TRUE-BDD is used as a basis. 

\textbf{Row-rules:}\\
We iterate over each row. For each row, an initially false BDD is built, \texttt{rowBDD}. Now, we iterate over the columns, producing a BDD, \texttt{ithVar} corresponding to a queen being placed in that column (in that row). When having placed a queen, we set the remainding spots in the row to false (making sure a queen cannot be placed within that row). This \texttt{ithVar} is now OR'ed onto the \texttt{rowBDD}. The argument for OR'ing these BDDs is used since a queen must be placed in either

First, the horisontal rules are added to the tree, followed by the vertical rules. 

\subsection{Placing a queen}
When placing a queen, we check whether the position (where the queen was supposed to be placed) is either already occupied or if a queen cannot be placed there. If neither, we continue, by placing a queen into that spot on the gameboard. \\

Continuing, the BDD is restricted, such that the variable in the BDD, that corresponds to the spot on the gameboard, is set true. \\

We make an assertion-test, that this manipulation of the BDD did not result in a non-satisfiable BDD. (This is a relic from implementation, that was used during debugging and development). 

Finally, the gameboard is updated, such that the gameboard reflects the constraints. \\

\texttt{updateGameBoard(...)} simply iterates over the variables in the BDD, and updates (if needed) the corresponding spot on the game board. This is done by checking whether restricting a given variable to true and false in turn, and checking whether one of these assignments results in an unsatisfiable configuration. If this is the case, the variable is assigned to the opposite value than the one that created the unsatisfiable configuration. In the case where neither results in an unsatisfiable configuration, we do not assign the value in the board. This is because this variable can be either value, based on the choice of the user. \\

Before \texttt{updateGameBoard(...)} iterates over each position, it checks whether the current configuration is satisfiable. If not, this means that there is no solution for the given configuration, and it therefore assigns crosses to all positions in the game board. This check runs every time a queen is placed, but is actually only required for the initial game board, because the logic above should restrict positions that makes the configuration unsatisfiable.

\section{Conclusion}
We believe we've solved the problem. We have tested the implementation, and to our beliefs it works. \\

We have checked that for game boards of size 2 and 3 there is no solution, and that some positions are unavailable for certain sizes of game boards.