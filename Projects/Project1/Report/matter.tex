\section{Introduction to implementation}
This report explains our implementation of the project. \\
Overall the project uses the classes seen in Figure \ref{fig:UMLClassDiagram}

\begin{figure}[h!]
\centering
\includegraphics[width=0.65\linewidth]{ClassDiagram.png}
\caption{UML class-diagram showing off classes implemented by group \label{fig:UMLClassDiagram}}
\end{figure}

The class \texttt{Utility9} is used to represent utility. The functions \texttt{maxValue} and \texttt{minValue} returns \texttt{Utility9} instances. It has two fields; \texttt{utiliity} to represent some utility, and a flag, \texttt{isTerminal} stating whether this is a terminal state or not. \\
Moreover, \texttt{Utility9} provides two static methods, \texttt{max(...)} and \texttt{min(...)}, that takes as arguments two \texttt{Utility9} instances, and returns the largest and the smallest, respectively; determined by comparing their utility-values. 

The \texttt{Gameboard9} class is used to encapsulate logic in relation to (the state of) and operations on the gameboard. To name a few, this includes a utility-function, a result-function and a function that returns the available actions.

Because the \texttt{GameBoard9} is used as the key in a cache for utility values, the class also overrides the \texttt{hashCode}-function and the equals function. These only compares the actual game board state, which internally is represented as a 2D array of integers.

\subsection{Evaluation and cut-off function}

Turning our attention to \texttt{GameLogic9}'s method, \texttt{decideNextMove()}, we initially extract our available actions, i.e. what columns we can fill in "coins". These actions are now explored within a set time-threshold (10 seconds) \textit{and} in increasing depth, untill time runs out. We continously search for a highest utility, \texttt{currentBest}; if the utility for the current action (in the current depth) exceeds the current highest utility, \texttt{currentBest} is updated. $\alpha$ is updated to whoever's largest of $\alpha$ (initially set to $-\infty$) and \texttt{utility}.


\subsection{Heuristic}
The heuristic used in the evaluation function is based on the fact that the more coins a player has in a row, column or diagonal the better position you are in. So For each coin you have in a line, a value is multiplied by 10, but if an opponent coin is placed anywhere in the line then the value is reduced to zero. That way only possible winning lines are rewarded. Furthermore, if a player has 3 in a row, and there is no coin on either left or right side, and the next move on those two columns are extending the line then a situation where victory is ensured has happened and the maximum possible value is returned.

\subsection{$\alpha$-$\beta$ pruning}
The \texttt{GameLogic9} class implements an $\alpha-\beta$ pruning search algorithm. The idea is to reduce the search space, to only look at values that aren't already guaranteed to be matched, for either of min-value and max-value. The order of which values are picked have a big inpact on how well alpha beta pruning works. If the best action is picked first for max nodes and the worst for min nodes we have an optimal case and the maximum amount of nodes are cut off. To approximate this optimal order of picking actions we use the heuristic of the resulting state of each action to order the actions. 

By doing this we improve the depth the gamelogic was able to go in from around 7 to 12 in the first move.